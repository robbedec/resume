\documentclass[10pt,a4paper]{article}

% NOTE: this only compiles with XeLaTeX (due to the fonts), because of the fonts.
% Also, because a package really wants a4paper apposed to letterpaper, we have to set a flag
% to keep compiling through warnings. ¯\_(ツ)_/¯
%
% So, to compiles this document, use the following shell command:
%
% xelatex -interaction=nonstopmode illya-starikov-resume.tex
%
% NOTE: This requires some fonts to compile properly, specially FontIn. You can find the font here: https://www.exljbris.com/fontin.html
% Place the fonts files (OTF)s at the same level as this document.

\usepackage{xunicode,xltxtra,url,parskip,enumitem,marvosym,graphicx,titlesec,hyperref,fontawesome,setspace,siunitx,multicol,textcomp}
\usepackage[usenames,dvipsnames]{xcolor}
\usepackage[big]{layaureo}
\usepackage[absolute]{textpos}
\usepackage{fontspec} % For loading fonts
\usepackage{ifthen}

\defaultfontfeatures{Mapping=tex-text}
\setmainfont[
    SmallCapsFont = Fontin-SmallCaps.otf,
    BoldFont =      Fontin-Bold.otf,
    ItalicFont =    Fontin-Italic.otf
]{Fontin-Regular.otf}
\setfontfamily{\FA}{[FontAwesome.otf]}
\defaultfontfeatures{Mapping=tex-text}

% Link color should be mostly black, with a border around it
\definecolor{linkcolour}{rgb}{0,0.2,0.6}
\hypersetup{colorlinks=false,allbordercolors={0 0 0},pdfborderstyle={/S/U/W 1}}

\newif\ifen
\newif\ifnl

% Just a useful utility for splitting up columns
\newcommand{\br}{\\\multicolumn{2}{c}{}}
\newcommand{\en}[1]{\ifen#1\fi}
\newcommand{\nl}[1]{\ifnl#1\fi}

% Set the margins to 5inches a piece
\newgeometry{noheadfoot=true,top=.5in,bottom=.5in, left=.5in, right=.5in}

% Make titles fit the page better
\titleformat{\section}{\Large\scshape\raggedright}{}{0em}{}[\titlerule]
\titlespacing{\section}{0pt}{3pt}{3pt}

% Make bullet point pretty
\setlist[itemize]{label=$\circ$, leftmargin=12pt, noitemsep, before={\vspace*{-.5\baselineskip}}, after={\vspace*{-\baselineskip}}}

% Don't hyphenate anything
\tolerance=1
\emergencystretch=\maxdimen%
\hyphenpenalty=10000
\hbadness=10000

% Absolute text positioning
\setlength{\TPHorizModule}{30mm}
\setlength{\TPVertModule}{\TPHorizModule}

% Don't hyphenate anything, make the text start closer to the ceiling
\textblockorigin{2mm}{0.65\paperheight}
\setlength{\parindent}{0pt}


% Define document language
% \entrue -> english / engels
% \nltrue -> dutch / nederlands
\nltrue


\begin{document}
\pagestyle{empty}

\par{\centering
    {\Huge \textsc{Robbe Decorte}
}\bigskip\par}

\vspace{-.5em}

\begin{multicols}{2}
\begin{tabular}{rl}
    \textsc{Email \faMailReplyAll}          & \href{mailto:robbe.decorte@ugent.be}{robbe.decorte@ugent.be} \\
    \textsc{\en{Phone}\nl{Gsm} \faPhone}    & +32 495 27 15 68 \\
    \textsc{LinkedIn \faLinkedin}           & \href{https://linkedin.com/in/robbedec/}{linkedin.com/in/robbedec} \\
    \textsc{Github \faGithub}               & \href{https://github.com/robbedec}{github.com/robbedec}
\end{tabular}

    \en{Engineer focused on computer science and information systems. Interested in computer vision and machine learning to improve everyday life. Enjoys sports and working on numerical problems.}
    
    \nl{Ingenieur (ing.) gefocust op computerwetenschappen en informaticasystemen. Geïnteresseerd in computervisie en machinaal leren om het dagelijks leven te verbeteren. Houdt zich graag bezig met sport en numerieke problemen. }

\end{multicols}

\vspace{-.5em}

\section{\en{Education}\nl{Onderwijs}}
\begin{tabular}{r|p{16cm}}
    \textsc{Jun 2022}      & \textbf{\en{(Master of Information Engineering Technology}\nl{Master in de industriële wetenschappen: informatica} (magna cum laude)} \\
    Sept 2021       & \textit{\en{Ghent University, Belgium}\nl{Universiteit Gent (UGent)}}  \\
    & Thesis: \en{Computer vision based scoring and evaluation of facial palsy patients}\nl{Computervisie gebaseerd scoren en evalueren van perifere aangezichtsverlamming}\\
    & \en{Relevant coursework}\nl{Relevante inhoud}:
    \en{Computer Vision}\nl{Computervisie},
    \en{Machine Learning}\nl{Machinaal leren},
    \en{Advanced Algorithms}\nl{Geavanceerde algoritmes},
    IoT,
    \en{Systems Design}\nl{Systeemontwerp},
    \en{System Administration}\nl{Systeembeheer},
    \en{Network and Computer Security}\nl{Beveiligen van netwerken en computers}
    \br\\
    
    \textsc{Sept 2021}      & \textbf{\en{(Linking Course Master of Information Engineering Technology}\nl{Schakelprogramma tot Master in de industriële wetenschappen: informatica} (cum laude)} \\
    Sept 2020       & \textit{\en{Ghent University, Belgium}\nl{Universiteit Gent (UGent)}}  \\
    & \en{Relevant coursework}\nl{Relevante inhoud}:
    \en{Mathematics}\nl{Wiskunde},
    \en{Statistics and Data Analysis}\nl{Statistiek en data-analyse},
    \en{Algorithms and Data Structures}\nl{Algoritmen en gegevensstructuren},
    \en{Computer Hardware}\nl{Computerhardware} (microprocessor, x86...),
    \en{Operating Systems}\nl{Besturingssystemen},
    \en{Computer Networks}\nl{Computernetwerken},
    \en{Signals and Systems}\nl{Signalen en systemen},
    \en{Databases}\nl{Databanken}
    \br\\
    
    \textsc{Jun 2020}      & \textbf{\en{Bachelor of Science, Applied Computer Science}\nl{Professionele Bachelor, Toegepaste Informatica} (cum laude)} \\
    Sept 2017       & \textit{\en{University College Ghent, Belgium}\nl{Hogeschool Gent (HoGent)}}  \\
    & \en{Relevant coursework}\nl{Relevante inhoud}:
    \en{Artificial Intelligence}\nl{Artificiële Intelligentie},
    \en{Object-Oriented Programming}\nl{Objectgeoriënteerd Programmeren},
    \en{Analysis of Algorithms}\nl{Analyseren van Algoritmes},
    \en{Statistics}\nl{Statistiek},
    Design Patterns,
    Linux \en{and}\nl{en} Windows Server,
    Database design,
    Web Development 
    \br\\
\end{tabular}

\section{\en{Technical Skills}\nl{Technische Skills}}
\begin{tabular}{r|p{16cm}}
    \textsc{\small \en{Languages}\nl{Talen}} & 
    \en{Dutch (native speaker)}\nl{Nederlands (moedertaal)},
    \en{English}\nl{Engels} (C1),
    \en{French (untested)}\nl{Frans (niet getest)} \br\\
    
    \textsc{\small Skills} &
    Python,
    C/C++,
    Java,
    Javascript,
    Typescript,
    SQL,
    \LaTeX{},
    Bash,
    Powershell,
    Kotlin,
    MATLAB
    \br\\
    
    \textsc{\small Tools} &
    NumPy,
    ScipPy,
    OpenCV,
    PyTorch,
    CUDA,
    Docker,
    Git,
    Linux,
    ASP.NET,
    Android,
    React.js,
    Spring boot,
    MQTT,
    Kafka,
    Vim
    \br\\
    
    \textsc{\small \en{Interests}\nl{Interesses}} &
    Machine Learning,
    Deep Learning,
    Tensorflow,
    \en{Evolutionary Algorithms}\nl{Genetische Algoritmes},
    Project Euler,
    IoT (Internet of Things)
    \br\\
\end{tabular}

\section{\en{Work Experience}\nl{Werkervaring}}
\begin{tabular}{r|p{16cm}}
    \textsc{Sept 2020}  & \textbf{\en{Software Engineering Internship (continued as a working student)}\nl{Stage Software Developer (voortgezet als jobstudent)}}  \\
    Feb 2020                         & \textit{In The Pocket}, \en{Ghent, Belgium}\nl{Gent}\\ &
    \begin{itemize}
        \item 
            \en{At In The Pocket, I belonged to a team that specialized in developing mobile applications using React Native and associated ecosystem.}
            \nl{Bij In The Pocket behoorde ik tot een team dat gespecialiseerd is in het ontwikkelen van mobiele applicaties m.b.v. React Native en bijhorend ecosysteem.}
        \item 
            \en{For the project in question, I work as a full team member with the accompanying responsibility of delivering complete and tested code in a timely manner.}
            \nl{Voor het desbetreffende project werk ik mee als volwaardig teamlid met de bijhorende verantwoordelijkheid van het tijdig opleveren van complete en geteste code.}
        \item 
            \en{I made contributions to the internal Hubble ecosystem that converts designer friendly formats to React components (or other JavaScript constructs). My focus here, was mainly on the conversion from Adobe XD.}
            \nl{Daarnaast heb ik bijdragen gemaakt aan het interne Hubble ecosysteem dat \textit{designer friendly} formaten omzet naar React componenten (of andere Javascript constructen). Mijn focus lag hier vooral op de conversie vanuit Adobe Xd.}
    \end{itemize} \br\\

    \textsc{\en{May}\nl{Mei} 2017} & \textbf{\en{Software Engineering Internship}\nl{Stage Software Developer}} \\
    Feb 2017           & \textit{\en{IT Department of Bruges City}\nl{IT Dienst Stad Brugge}}, \en{Bruges, Belgium}\nl{Brugge} \\ &
    \begin{itemize}
        \item 
            \en{Improving accessibility of an in-house developed web application by the city of Bruges. My task was mostly focussed on blind people and those who are visually impaired.}
            \nl{Toegankelijk maken van een ontwikkelde webapplicatie door Stad Brugge voor blinden en slechtzienden, gebruik maken van Aria en gelijkaardige libraries die het gebruik van screenreaders mogelijk maken.}
        \item 
            \en{Updating database documentation and writing queries (Linq and SQL) that feed a visualisation tool.}
            \nl{Updaten van databank documentatie en het schrijven van queries (Linq en SQL) die data feeden aan een visualisatie tool.}
        \item 
            \en{Taking part in meetings and the daily Stand-Up. Supporting the SCRUM-master and how to start and finish a sprint.}
            \nl{Bijwonen van teammeetings en dagelijkse Stand-Up. Helpen bij de taken van de SCRUM-master en het opzetten / afsluiten van de sprint.}
    \end{itemize} \br\\
\end{tabular}
\end{document}
